\chapter{Introduction}

\section{Background and Historical Context}

Arsenic pollution in groundwater is a problem faced worldwide \parencite{khan2014groundwater}. Southeast Asian populations, particularly those of Bangladesh, are arguably the most negatively impacted by arsenic contamination in drinking water worldwide. 

 \parencite{Smith2000}. 

These tube wells were typically installed at a depth of under 200 meters. Arsenic concentrations in groundwater peak at between 15 and 25 meters of depth and are typically only below the World Health Organisation's safe arsenic limit of 10$\mu$g/litre \parencite{WHO2003} at 200 meters of depth or more \parencite{Chakraborti2010}. 

Prior to the discovery of harmful levels of arsenic in the water from these tube wells, their installation was acclaimed as a huge success with regards to providing the population with clean drinking water \parencite{khan2014groundwater}. Tubewells today are still extremely popular in Bangladesh with approximately 95\% of rural people and 70\% of urban people using them in 2020 \parencite{Ghosh2020}.

\subsection{Solution Approach}

\cite{Smith2000} emphasizes that the core of any solution must be to provide non-polluted water.

This article makes it clear that there is no single solution to this issue, it may seem simple to provide water filters, for example, but for their successful deployment, they must be provided with education and community engagement to ensure their efficacy. They are also not a permanent solution. 

\cite{Khan2014} illustrates the scale and complexity of a comprehensive solution approach by analysing the motivations of key stakeholders.

\section{What is iArsenic}

iArsenic is a web application that estimates the arsenic concentration for a well in Bangladesh based on its geographic region, depth and visible staining on the well.

This is achieved by aggregating source data about these wells and using that aggregate data to produce expert system prediction models which are available via a web application.

The tooling developed to produce these models are general purpose and modular, enabling them to be imported into other projects. The time and expertise required to clean and aggregate the source data used by these models are very substantial and leveraging this existing work allows higher level outcomes to be brought into scope.

The data in iArsenic has been selected from reputable sources and represents a large and high quality dataset, representing the majority of Bangladesh.

\section{Research Questions}
\label{RQ}

1. What is the performance of the existing iArsenic models.

2. Do machine learning models have potential in the prediction of groundwater arsenic pollution?

\section{Conclusion}

Arsenic pollution is a serious and ongoing issue and it is a great shame that despite relevant organisations being aware of this problem since at least the year 2000 it persists to today in 2023.

This project does not aim to conquer and eliminate this problem, but does aim to make a contribution to the ongoing research into the use of predictive models to predict groundwater arsenic pollution.

To achieve this contribution, this project aims to validate existing predictive models and provide an example for the application of machine learning in this field.