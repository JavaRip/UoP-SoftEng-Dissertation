\chapter{Reflection \& Conclusion}

\section{Research Questions}

\subsection{What is the performance of the existing iArsenic models}

Table \ref{tbl:x model_evals} on page \pageref{tbl:x model_evals} shows a thorough analysis of the existing iArsenic models.

These models are deployed in the iArsenic web application and this evaluation provides a key and valuable contribution which was not available before.

This project showed that the performance of the deployed iArsenic model, model5, performs at least as well or better than other models examined in the literature review.

\subsection{Do machine learning models have potential in the prediction of groundwater arsenic pollution?}

While none of the machine learning models outperformed the iArsenic models, model6, model13 and model14 each achieved an f1 score of 58\% or higher.

% TODO reference loss curve in future work
While model6, a random forest classifier and model13, a k-nearest neighbours classifier may be reaching their full potential in this implementation, the feed forward neural network classifiers show potential for further performance improvements with more computing power, optimization or models based on new frameworks. See \ref{fannm} on page \pageref{fannm} and \ref{cnnbm} on page \pageref{cnnbm}.

Despite not outperforming the expert system models, the machine learning models achieved a performance level comparable to the iArsenic models and the other predictive models in the literature review. Therefore the answer to this research question is yes, machine learning models can predict groundwater arsenic pollution, and it is possible that with further work and potentially the use of convolutional neural networks, the performance of the iArsenic models will be exceeded.

\section{New Models Development Strategy}

When developing new models a large portion of time was spent refining the models. While refinement is important and that without refinement the machine learning models would not have performed as well, much of the time spent refining could have been spent exploring different model types.

I was personally driven by a desire to exceed the performance of the existing models, which was at no point a requirement from the client or a requirement to answer the research questions. This has led to fewer model types being explored and potentially viable model types being undiscovered.

If I was to do this project again, I would focus on developing more and a more diverse range of model types.

\section{Achievement of Personal Goals}

The selection of this project for the dissertation was influenced by my own personal motivations. This project provided an amazing opportunity to develop machine learning and artificial intelligence skills and develop experience in the field of improving access to clean drinking water. This has made working on this project an immense privilege.

In some ways, I am left feeling frustrated at the decisions I made throughout the development of this project which resulted in dead ends, wastes of time and moments where I was stuck and unable to make progress for reasons that now seem so obvious to overcome. Ultimately though this frustration stems from the wisdom I now have which I did not have at the start of this project and that represents a huge success.

Having convolutional neural networks next on the roadmap of the development of my skills simply feels unbelievable, as I feel the opportunity to have a working knowledge of cutting edge artificial intelligence and machine learning methodology is now achievable.

Developing this understanding of machine learning and working on a good, worthwhile cause has been immensely personally fulfilling and I am grateful for the opportunities this project has presented.

\section{Conclusion}

This project has been a success, answering the research questions via the desired outcomes and making a genuine contribution to an existing project, iArsenic.

There is more work to do in this project and to the benefit of iArsenic more generally, including some exciting opportunities in deep learning, with this project providing the foundation for this further work.