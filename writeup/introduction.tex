\chapter{Introduction}

\section{Background and Historical Context}

Arsenic pollution in groundwater is a problem faced worldwide \parencite{khan2014groundwater}. Southeast Asian populations, particularly those of Bangladesh, are arguably the most negatively impacted by arsenic contamination in drinking water worldwide. 

This is largely due to the use of tube wells, many of which were built as a result of the United Nations Children's Fund (UNICEF) working with the Bangladeshi government from 1970 to provide safe drinking water to 80\% of the population by the year 2000 \parencite{Smith2000}. 

These tube wells were typically installed at a depth of under 200 meters. Arsenic concentrations in groundwater peak at between 15 and 25 meters of depth and are typically only below the World Health Organisation's safe arsenic limit of 10$\mu$g/litre \parencite{WHO2003} at 200 meters of depth or more \parencite{Chakraborti2010}. 

Prior to the discovery of harmful levels of arsenic in the water from these tube wells, their installation was acclaimed as a huge success with regards to providing the population with clean drinking water \parencite{khan2014groundwater}. Tubewells today are still extremely popular in Bangladesh with approximately 95\% of rural people and 70\% of urban people using them in 2020 \parencite{Ghosh2020}.

\section{Solution Approach}

- The exact nature of this problem is highly complex

- To overcome the issue, there is no one solution, rather, multiple solutions which each tackle an aspect of the problem

Sub issues to tackle

- providing access to safe drinking water

- Education about drinking water and the negative health impact of polluted water, often people find it hard to believe water that appears clear can be dangerous / linked to disease

- Medical treatment for those affected (just those who have developed disease such as cancer or those with acute arsenic poisining? Find citations arguing those points)

- Policy and government changes such as aiming for comprehensive testing of wells and the disabling of polluted wells and provision of an alternative source, where there is no source it would only cost \$x to deepen the well to a safe level

- Humanitarian engagement with communities to ensure cooperation with practices designed to mitigate exposure (ie not using a well that's been confirmed as polluted) 

\section{What is iArsenic}

iArsenic is a web application that estimates the arsenic concentration for a well in Bangladesh based on its geographic region, depth and visible staining on the well.

This is achieved by aggregating source data about these wells and using that aggregate data to produce expert system prediction models which are available on the web page.

The tooling developed to produce these models are general-purpose and modular, enabling them to be imported into other projects. The time and expertise required to clean and aggregate the source data are very substantial and leveraging this existing work allows higher-level outcomes to be brought into scope.

The data in iArsenic has been selected from reputable sources and represents a large and high-quality dataset, representing the majority of Bangladesh.