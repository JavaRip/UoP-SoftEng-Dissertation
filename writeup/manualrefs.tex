\chapter{References}

Bindal, S., \& Singh, C. K. (2019). Predicting groundwater arsenic contamination: Regions at risk in highest populated state of India. Water research, 159, 65-76.\\

Caruana, R., \& Niculescu-Mizil, A. (2006, June). An empirical comparison of supervised learning algorithms. In Proceedings of the 23rd international conference on Machine learning (pp. 161-168).\\

Castelvecchi, D. (2016). Can we open the black box of AI?. Nature News, 538(7623), 20.\\

Connolly, C. T., Stahl, M. O., DeYoung, B. A., \& Bostick, B. C. (2021). Surface flooding as a key driver of groundwater arsenic contamination in Southeast Asia. Environmental Science \& Technology, 56(2), 928-937.\\

Chakraborti, D., Rahman, M. M., Das, B., Murrill, M., Dey, S., Mukherjee, S. C., ... \& Quamruzzaman, Q. (2010). Status of groundwater arsenic contamination in Bangladesh: a 14-year study report. Water research, 44(19), 5789-5802.\\

Connolly, C. T., Stahl, M. O., DeYoung, B. A., \& Bostick, B. C. (2021). Surface flooding as a key driver of groundwater arsenic contamination in Southeast Asia. Environmental Science \& Technology, 56(2), 928-937.\\

Fleming, S. W., Watson, J. R., Ellenson, A., Cannon, A. J., \& Vesselinov, V. C. (2021). Machine learning in Earth and environmental science requires education and research policy reforms. Nature Geoscience, 14(12), 878-880.\\

Géron, A. (2017). Hands-on machine learning with scikit-learn and tensorflow: Concepts. Tools, and Techniques to build intelligent systems.\\

Ghosh, G. C., Khan, M., Hassan, J., Chakraborty, T. K., Zaman, S., Kabir, A. H. M., \& Tanaka, H. (2020). Human health risk assessment of elevated and variable iron and manganese intake with arsenic-safe groundwater in Jashore, Bangladesh. Scientific reports, 10(1), 1-9.\\

Guidotti, R., Monreale, A., Ruggieri, S., Turini, F., Giannotti, F., \& Pedreschi, D. (2018). A survey of methods for explaining black box models. ACM computing surveys (CSUR), 51(5), 1-42.\\

Khan, N. I., \& Yang, H. (2014). Arsenic mitigation in Bangladesh: An analysis of institutional stakeholders' opinions. Science of the total environment, 488, 493-504.\\

Loyola-Gonzalez, O. (2019). Black-box vs. white-box: Understanding their advantages and weaknesses from a practical point of view. IEEE Access, 7, 154096-154113.\\

Najafabadi, M. M., Villanustre, F., Khoshgoftaar, T. M., Seliya, N., Wald, R., \& Muharemagic, E. (2015). Deep learning applications and challenges in big data analytics. Journal of big data, 2(1), 1-21.\\

Smith, A. H., Lingas, E. O., \& Rahman, M. (2000). Contamination of drinking-water by arsenic in Bangladesh: a public health emergency. Bulletin of the World Health Organization, 78(9), 1093-1103.\\

Van Halem, D., Bakker, S. A., Amy, G. L., \& Van Dijk, J. C. (2009). Arsenic in drinking water: a worldwide water quality concern for water supply companies. Drinking Water Engineering and Science, 2(1), 29-34.\\

Winkel, L., Berg, M., Amini, M., Hug, S. J., \& Annette Johnson, C. (2008). Predicting groundwater arsenic contamination in Southeast Asia from surface parameters. Nature Geoscience, 1(8), 536-542.\\

World Health Organization. (2003). Arsenic in drinking-water: background document for development of WHO guidelines for drinking-water quality (No. WHO/SDE/WSH/03.04/75). World Health Organization.\\